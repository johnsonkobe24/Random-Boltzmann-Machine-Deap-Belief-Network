\section*{START HERE: Instructions}

\begin{notebox}
Homework 2 covers topics on graphical models. 
The homework includes short answer questions, derivation questions, and a coding task. 
\end{notebox}

\begin{itemize}
    \item \textbf{Collaboration policy:} Collaboration on solving the homework is allowed, after you have thought about the problems on your own. It is also OK to get clarification (but not solutions) from books or online resources, again after you have thought about the problems on your own. There are two requirements: first, cite your collaborators fully and completely (e.g., ``Jane explained to me what is asked in Question 2.1''). Second, write your solution {\em independently}: close the book and all of your notes, and send collaborators out of the room, so that the solution comes from you only.  
    See the Academic Integrity Section on the course site for more information: \url{https://deeplearning-cmu-10707-2022spring.github.io/index.html#policies}
    
    \item\textbf{Late Submission Policy:} See the late submission policy here:\\
    \url{https://deeplearning-cmu-10707-2022spring.github.io/index.html#policies}
    
    \item\textbf{Submitting your work:} 

    \begin{itemize}
        \item \textbf{Gradescope:} For written problems such as short answer, multiple choice, derivations, proofs, or plots, we will be using Gradescope (\url{https://gradescope.com/}).
        Please write your solution in the LaTeX files provided in the assignment and submit in a PDF form. Put your answers in the question boxes (between \texttt{\textbackslash begin\{soln\}} and \texttt{\textbackslash end\{soln\}}) below each problem. Please make sure you complete your answers within the given size of the question boxes. \textbf{Handwritten solutions are not accepted and will receive zero credit.} Regrade requests can be made, however this gives the TA the opportunity to regrade your entire paper, meaning if additional mistakes are found then points will be deducted. For more information about how to submit your assignment, see the following tutorial (note that even though the assignment in the tutorial is handwritten, submissions must be typed): \url{https://www.youtube.com/watch?v=KMPoby5g_nE&feature=youtu.be}

        \item \textbf{Code submission:} All code must be submitted to a Gradescope autograder named as ``Assignment 2: Programming''.  \textbf{If you do not submit your code here, you will not receive any credit for your assignment.} Gradescope grader will be used to check for plagiarism.  Please make sure you familiarize yourself with the academic integrity information for this course.
    \end{itemize}

\end{itemize}

\textbf{Important Notes on the Written Problems}:
\begin{itemize}
    \item Please make sure that the solution boxes do not move from the original places in your writeup pdf. And also please check whether your solution boxes fit into the solution areas when you submit your pdf to Gradescope. 
    \item If you have more text than the solution box, you can resize the box using height option in the latex command $\backslash begin\{soln\}\{height=10cm\}$.
    \item You can also put $\backslash pagebreak$ before the solution environment (before $\backslash begin\{soln\}$) to make your final pdf look better-organized.
\end{itemize}

\textbf{Important Notes on the Programming Problems}:
\begin{itemize}
    \item Do not use any toolboxes except those already imported in the code template. 
    \item Read the doc-strings/comments in the template very carefully before you start. 
    \item Reach out for help on Piazza or during office hours when you struggle. 
    \item Do not change any function signatures because your code will be auto-graded. 
    \item Try to vectorize the computation as much as possible (e.g. compute in the form of matrix multiplication, utilize numpy functions instead of loops, etc.)
    \item Use Python 3.6 or above, and the latest version of numpy.
\end{itemize}